\chapter*{TÓM TẮT ĐỒ ÁN}
\addcontentsline{toc}{chapter}{TÓM TẮT ĐỒ ÁN}

Agobot, thường được biết đến là Gaobot, là một họ sâu máy tính. Lập trình viên
người Đức - Axel Ago Gembe - là người chịu trách nhiệm viết phiên bản đầu tiên của
Agobot. Mã nguồn Agobot mô tả nó là: "bot IRC dạng mô-đun dành cho Win32/Linux".
Agobot được phát hành theo giấy phép GPL2. Agobot là chương trình đa luồng và hướng
đối tượng được viết bằng C++, trong khi các thành phần nhỏ hơn được viết bằng hợp
ngữ. Agobot là một ví dụ về botnet mà hầu như không hoặc yêu cầu rất ít kiến thức lập
trình để có thể sử dụng nó.

Agobot (hay còn gọi là Gaobot, Polybot, và Phatbot) trình bày chính nó như là một
nghiên cứu thú vị vì nó có thể là virus được lưu hành rộng rãi nhất trong lịch sử, chỉ dựa
trên số lượng các biến thể duy nhất được tạo ra. Số lượng các phiên bản hiện tại là
khoảng 1200, nhưng không thể biết được có bao nhiêu người làm bản sao có thể đã biên
dịch bản sao virus của riêng họ. Số lượng các biến thể là do sự sẵn có của mã nguồn được
công bố, bị cáo buộc được viết bởi tác giả Axel "Ago" Gembe bị bắt vào tháng 5 năm
2004 tại Đức. Số lượng biến thể khổng lồ này làm cho Agobot trở thành một thách thức
để kiểm tra. Việc sử dụng các biến thể kết hợp với các mảng hoán vị tính toán vô hạn gần
như buộc phải bắt đầu bằng cách chỉ nhìn vào một vài khía cạnh của chương trình.

Agobot không phù hợp với mô hình chuẩn của "virus kit", vì nó không bị giới hạn
bởi một tập các biến được sắp xếp trong bộ tạo worm. Trong thực tế, nó vượt ra ngoài bộ
virus tiêu chuẩn bằng cách cung cấp một trình bao cho bất kỳ số lần khai thác nào, không
chỉ những người được phát hiện tại thời điểm tạo ra Agobot. Bài báo cáo này sẽ giới thiệu
các bộ virus, xác định vị trí của Agobot trong lịch sử của virus kit, và chi tiết một số biến
thể và cách chúng hoạt động. Tham chiếu chung cho các thuật ngữ "Agobot" và
"Gaobot" đại diện cho toàn bộ gia đình các biến thể của nó, bao gồm cả Agobot gốc
thông qua hậu duệ của nó, Phatbot.

Bài viết này đánh giá mối đe dọa của các biến thể có nguồn gốc từ Agobot bằng cách
kiểm tra sự phát triển của virus, các bản release của source code, và một vài lần lặp lại cụ
thể. Phân tích này đặt mã Agobot trong danh mục “virus kit”. Từ phân loại này, Agobot
được trình bày có thể là bộ virus kit thành công nhất trong lịch sử, không phải vì số lượng
biến thể hoặc máy chủ bị nhiễm, mà là vì điều chỉnh trong phòng chống virus nó đã yêu
cầu.

% Nội dung gồm 2 phần chính:

% \begin{itemize}
% \item asdf
% \end{itemize}

\clearpage

\chapter*{MỞ ĐẦU}
\addcontentsline{toc}{chapter}{MỞ ĐẦU}

\section*{Lý do chọn đề tài}
Trong cuộc sống hàng ngày của chúng ta gia tăng sự phụ thuộc vào Internet, thêm
vào đó là rất nhiều thách thức trong việc quản lý Internet và sử dụng các ứng dụng như
là: bảo vệ, an toàn, toàn vẹn và khả dụng dữ liệu người dùng. Trong những năm gần đây,
Intenet đóng một vai trò chính trong cuộc sống của chúng ta, đặc biệt trong truyền thông,
giáo dục, dịch vụ công, ngân hàng và thương mại... Không may mắn là việc gia tăng nhu
cầu sử dụng các ứng dụng trở thành mối đe dọa đến tính riêng tư và an toàn dữ liệu của
người dùng. Botnet là mộ phần mềm điều khiển máy tính với mục đích độc hại, gọi là
bots. Bots là những kịch bản nhỏ đã được xây dựng để thực hiện những tác vụ tự động và
đặc biệt. Những bots này được điều khiển bởi một hay một nhóm nhỏ hackers gọi là
"botmaster". Theo thống kê của McAfee Labs, số phần mềm độc hại mới được phát hiện
đạt tới 50 triệu trong quý 4 năm 2014 và có thể đạt tới nửa tỷ vào cuối năm 2015. Hơn
nữa, lưu lượng Internet bao gồm đến 80\% là lưu lượng botnets liên quan đến những email
spam có nguồn gốc từ những botnet nổi tiếng như: Grum, Cutwail và Rustock. Ngày nay,
những botnets quy mô lớn có thể tung ra nhiều hơn 1 triệu PCs để tấn công không gian
mạng.

Những botnets khác nhau từ các loại phần mềm độc hai thông qua việc sử dụng một
kênh truyền thông để nhận lệnh và thông báo trạng thái hiện hành tới người điều hành.
Theo báo cáo của Golden State năm 2014, cuộc tấn công không gian mạng vào đêm
Giáng sinh vào website ngân hàng vùng California đã làm ngưng trệ công việc của các
nhân viên trong việc phục hồi một tài khoản trong số các khách hàng của họ và đã bị mất
hơn 900,000 đô la. Thêm nữa, năm 2013 FBI thông báo 10 hacker quốc tế đã bị bắt giữ
khi sử dụng botnet để trộm hơn 850 triệu đô la thông qua một hệ thống máy tính bị lây
nhiễm. Chúng đã sử dụng thông tin tài chính cá nhân của mọi người để trộm số tiền như
vậy.

Trong thực tế, botnet có những đặc tính riêng biệt so với các kiểu phần mềm độc hại
khác. Ví dụ, botmaster có điều khiển những máy tính bị nhiễm và gửi lệnh cho nó mà
không cần trực tiếp liên lạc với chúng. Cũng có nhiều bots làm việc theo cách phối hợp
và nhận hướng dẫn từ botmaster để nhanh chóng phối hợp tấn công như tấn công từ chối
dịch vị phân tán, spam phân tán và lừa đảo. Ngoài ra, botnet còn cung cấp các gian lận
này như một hình thức dịch vụ được xem xét là một phần của vấn đề botnet tấn công vào
nền kinh tế.

Chính vì những lí do đó mà nhóm chúng tôi đã thực hiện đề tài này.

\section*{Mục đích thực hiện đề tài}
Khi chọn thực hiện đề tài, nhóm chúng tôi mong muốn được tìm hiểu và tiếp cận các
phương pháp phân tích mã độc, từ đó vận dụng vào ngành của mình đang học – ngành
An toàn thông tin, và các công việc sau này.

Các mục tiêu hướng đến là:

\begin{itemize}
\item Làm quen, tiếp cận các phương pháp mã độc từ trước đến nay.
\item Phân loại rõ sự khác biệt giữa Agobot và các loại botnet khác, và giữa các biến
thể của các botnet của chính Agobot.
\end{itemize}

\section*{Đối tượng và phạm vi nghiên cứu của đề tài}
Đối tượng và phạm vi nghiên cứu tập trung vào:

\begin{itemize}
\item Agobot nói chung và các biến thể của Agobot nói riêng.
\end{itemize}

Trong phạm vi của đồ án, nhóm chỉ tập trung vào phân tích phiên bản gốc Agobot là
chính.
