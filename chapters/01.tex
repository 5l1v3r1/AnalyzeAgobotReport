% \pagebreak[4]
% \hspace*{1cm}
% \pagebreak[4]
% \hspace*{1cm}
% \pagebreak[4]

\chapter{Phân tích codebase}\label{ch:codebase}

\section{Cấu trúc mã nguồn}

Cấu trúc mã nguồn đề cập đến các đặc điểm thiết kế và cài đặt của mã bot.
Agobot bao gồm 20.000 dòng C++ với tính tương thích trên nhiều nền tảng và
được phát hành giấy phép mã nguồn mở GPL\@.

Agobot được viết bởi Ago với bí danh Wonk,
thanh niên người Đức bị bắt vào tháng 5 năm 2004 vì tội phạm mạng.
Các phiên bản mới nhất có của Agobot có thiết kế trừu tượng cấp cao với
nhiều thành phần cấp cao bao gồm~\cite{inside,honeynet53}:

\begin{itemize}
\item Cơ chế điều khiển và chỉ huy dựa trên IRC\@.
\item Lượng lớn các mã độc dùng để khai thác mục tiêu.
\item Khả năng thực hiện các kiểu tấn công DoS khác nhau.
\item Thành phần hỗ trợ hỗ trợ mã hóa shellcode và thành phần cơ chế làm rối cơ bản dựa trên đa hình.
\item Khả năng thu thập mật khẩu Paypal,
	khoá phần mềm AOL và các thông tin nhạy cảm khác từ máy bị lây nhiễm thông qua bắt gói tin,
	ghi lại sự kiện bàn phím (key logger) hoặc tìm kiếm các khoá Windows Registry.
\item Cơ chế bảo vệ và tăng quyền kiểm soát các hệ thống bị chiếm đoạt sẵn bằng cách đóng các backdoor,
	và vá các lổ hỗng khác hoặc vô hiệu truy cập vào các trang web chống virus.
\item Cơ chế chống gỡ lỗi và dịch ngược (anti-debugging) bởi các công cụ nổi tiếng như
	SoftIce, OllyDbg,~\dots~Cơ chế nhận diện ảo hoá trong máy ảo (ví dụ: VMWare và Virtual PC).
\end{itemize}

Agobot có kiến trúc nguyên khối,
thể hiện tính sáng tạo trong thiết kế và tuân thủ nguyên tắc thiết kế với tập các cấu trúc và
kiểu dữ liệu mở rộng được kiểm soát chặt chẽ và nguyên tắc kỹ thuật phần mềm thông qua tính
mô-đun hoá và tài liệu mã nguồn mạnh mẽ.

Agobot được cấu trúc theo cách rất mô đun và rất dễ dàng để thêm các lệnh
hoặc máy quét cho các lỗ hổng khác:
Chỉ bằng cách cần mở rộng lớp \texttt{CCommandHandler} hoặc
\texttt{CScanner} và thêm tính năng tương ứng.
Một mẹo nhỏ~\cite{honeynetgrep}:
Chỉ cần dùng lệnh grep mã nguồn với từ khoá \texttt{RegisterCommand},
ta sẽ có được toàn bộ danh sách lệnh với một mô tả đầy đủ của tất cả các tính năng.

\section{Cơ chế điều khiển từ xa}

Phương thức điều khiển từ xa gồm các ngôn ngữ chỉ huy và giao thức điều khiển được sử dụng
để vận hành botnet từ xa sau khi hệ thống đích đã bị xâm phạm.

Lý do quan trọng nhất để đi tìm hiểu chi tiết về phương thức
giao tiếp của botnet là sự gián đoạn của nó có thể khiến cho botnet vô dụng.
Ví dụ, nhà cung cấp mạng có thể xác định các hệ thống bị xâm phạm bằng cách
tìm kiếm các lệnh botnet cụ thể trong lưu lượng IRC, và các nhà cung cấp
máy chủ IRC có thể tắt các kênh (channel) được sử dụng bởi botnet.

Agobot có hệ thống lệnh và điều khiển là một trích xuất của IRC\@.
Giao thức được dùng để thiếp lập kết nối giữa hệ thống bị xâm phạm đến kênh điều khiển là IRC chuẩn.
Còn ngôn ngữ lệnh gồm cách lệnh IRC chuẩn và các lệnh dành riêng của nó.

\begin{table}[ht]
	\caption{Các ``biến'' làm tham số cho lệnh \texttt{cvar.set}.}\label{table:cvar_set}
	\centering
	\footnotesize
	\begin{tabular}{ll}\toprule
		\textbf{Variable} & \textbf{Description}\\\midrule
		\verb|bot_ftrans_port| & Set bot - file transfer port\\
		\verb|bot_ftrans_port_ftp| & Set bot - file transfer port for FTP\\
		\verb|si_chanpass| & IRC server information - channel password\\
		\verb|si_mainchan| & IRC server information - main channel\\
		\verb|si_nickprefix| & IRC server information - nickname prefix\\
		\verb|si_port| & IRC server information - server port\\
		\verb|si_server| & IRC server information - server address\\
		\verb|si_servpass| & IRC server information - server password\\
		\verb|si_usessl| & IRC server information - use SSL?\\
		\verb|si_nick| & IRC server information - nickname\\
		\verb|bot_version| & Bot - version\\
		\verb|bot_filename| & Bot - runtime filename\\
		\verb|bot_id| & Bot - current ID\\
		\verb|bot_prefix| & Bot - command prefix\\
		\verb|bot_timeo| & Bot - timeout for receiving (in milliseconds)\\
		\verb|bot_seclogin| & Bot - enable login only by channel messages\\
		\verb|bot_compnick| & Bot - use the computer name as a nickname\\
		\verb|bot_randnick| & Bot - random nicknames of letters and numbers\\
		\verb|bot_meltserver| & Bot - melt the original server file\\
		\verb|bot_topiccmd| & Bot - execute topic commands\\
		\verb|do_speedtest| & Bot - do speed test on startup\\
		\verb|do_avkill| & Bot - enable anti-virus kill\\
		\verb|do_stealth| & Bot - enable stealth operation\\
		\verb|as_valname| & Autostart - value name\\
		\verb|as_enabled| & Autostart - enabled\\
		\verb|as_service| & Autostart - start as service\\
		\verb|as_service_name| & Autostart - short service name\\
		\verb|scan_maxthreads| & Scanner - maximum number of threads\\
		\verb|scan_maxsockets| & Scanner - Maximum number of sockets\\
		\verb|ddos_maxthreads| & DDoS - maximum number of threads\\
		\verb|redir_maxthreads| & Redirect - maximum number of threads\\
		\verb|identd_enabled| & IdentD - enable the server\\
		\verb|cdkey_windows| & Return windows product keys on cdkey.get\\
		\verb|scaninfo_chan| & Scanner - output channel\\
		\verb|scaninfo_level| & Info level 1 (less) - (3) more\\
		\verb|spam_aol_channel| & AOL spam - channel name\\
		\verb|spam_aol_enabled| & AOL spam - enabled?\\
		\verb|sniffer_enabled| & Sniffer - enabled?\\
		\verb|sniffer_channel| & Sniffer - output channel\\
		\verb|vuln_channel| & Vulnerability daemon sniffer channel\\
		\verb|inst_polymorph| & Installer - polymorphoic on install?\\\bottomrule
	\end{tabular}
\end{table}

\begin{table}[ht]
	\caption{Một phần ngôn ngữ lệnh và điều khiển của Agobot.}\label{table:botcmd}
	\centering
	\footnotesize
	\begin{tabular}{l l}
		\toprule
		\textbf{Command} &
		\textbf{Description} \\
		\midrule
		\texttt{bot.about} & Displays information (e.g., version) about the bot code \\
		\texttt{bot.die} & Terminates the bot \\
		\texttt{bot.dns} & Resolves IP/hostname via DNS \\
		\texttt{bot.execute} & Makes the bot execute a specific~.exe \\
		\texttt{bot.id} & Displays the ID of the current bot code \\
		\texttt{bot.nick} & Changes the nickname of the bot \\
		\texttt{bot.open} & Opens a specified file \\
		\texttt{bot.remove} & Removes the bot from the host \\
		\texttt{bot.removeallbut} & Removes the bot if ID does not match \\
		\texttt{bot.rndnick} & Makes the bot generate a new random nickname \\
		\texttt{bot.status} & Echo bot status information \\
		\texttt{bot.sysinfo} & Echo the bot's system information \\
		\texttt{bot.longuptime} & If uptime > 7 days then bot will respond \\
		\texttt{bot.highspeed} & If speed > 5000 then bot will respond \\
		\texttt{bot.quit} & Quits the bot \\
		\texttt{bot.flushdns} & Flushes the bot's DNS cache \\
		\texttt{bot.secure} & Delete specified shares and disable DCOM \\
		\texttt{bot.unsecure} & Enable specified shares and enables DCOM \\
		\texttt{bot.command} & Executes a specified command with \verb|system()| \\
		\bottomrule
	\end{tabular}
\end{table}

Một phần ngôn ngữ lệnh được thể hiện trong Bảng~\ref{table:botcmd}
(tham khảo~\cite{lurhq} để xem danh sách lệnh đầy đủ hơn).
Tập lệnh bao gồm các chỉ thị yêu cầu bot thực hiện một chức năng cụ thể,
ví dụ: \texttt{bot.open} mở một tệp cụ thể trên máy chủ.
Các biến điều khiển (Bảng~\ref{table:cvar_set}) được sử dụng trong lệnh \texttt{cvar.set}
để bật/tắt các tính năng hoặc thao tác trường ảnh hưởng đến phương thức hoạt động,
ví dụ:
\verb|ddos_max_threads| thiết đặt số luồng tối đa dùng để làm ngập máy chủ cụ thể.

Xem đoạn mã~\ref{code:cvar_set} và đoạn mã~\ref{code:botcmd} để tìm hiểu thêm về các lệnh điều khiển từ xa.

Một ví dụ về lệnh \texttt{bot.command} (Đoạn mã~\ref{code:bot_command}) --
lệnh này dùng để thực thi môt lệnh tuỳ ý với hàm \texttt{system()} (dòng 14).
Nếu \texttt{system} trả về giá trị khác $-1$,
hàm sẽ dùng IRC để thông báo về việc thực thi thành công.

\begin{listing}[ht]
\caption{Đoạn code định nghĩa các biến để điều khiển bot (thuộc file \textbf{bot.cpp})}\label{code:cvar_set}
\inputminted{cpp}{./listings/var_bot.cpp}
\end{listing}

\begin{listing}[ht]
\caption{Các lệnh điều khiển từ xa trong mã nguồn Phatbot (thuộc file \textbf{bot.cpp})}\label{code:botcmd}
\inputminted{cpp}{./listings/botcmd_bot.cpp}
\end{listing}

\begin{listing}[ht]
\caption{Một ví dụ về lệnh \texttt{bot.command} (trong hàm \texttt{CBot::HandleCommand} thuộc file \textbf{bot.cpp})}\label{code:bot_command}
\inputminted[
	fontsize=\footnotesize,
	xleftmargin=20pt,
	linenos,
]{cpp}{./listings/bot_command-bot.cpp}
\end{listing}

\textbf{Nhận xét}:
Chúng tôi dự đoán rằng Agobot trong tương lai có thể bao gồm việc sử dụng giao tiếp được mã hóa,
chuyển từ IRC sang mô hình mạng ngang hàng (peer-to-peer).
Trong khi điều này chắc chắn sẽ khiến việc chống lại botnet khó khăn hơn,
nhưng đa phần lưu lượng botnet vẫn có thể được xác định thông qua phương pháp thống kê dấu vân tay.

\section{Điều khiển máy chủ}

Điều khiển máy chủ bao gồm các cơ chế được bot sử dụng để điều khiển máy chủ nạn nhân khi nó đã bị xâm phạm. Mục đích chung của kiểm soát máy chủ là củng cố quyền điều khiển máy nạn nhân bằng
cách chống các tấn công nguy hiểm khác,
để vô hiệu hóa phần mềm chống vi-rút và thu thập thông tin nhạy cảm.

\begin{table}[ht]
	\caption{Ngôn ngữ điều khiển máy chủ Agobot.}\label{table:botctrl}
	\centering
	\footnotesize
	\begin{tabular}{l l}
		\toprule
		\textbf{Command} &
		\textbf{Description} \\
		\midrule
		\texttt{harvest.cdkeys} & Return a list of CD keys \\
		\texttt{harvest.emails} & Return a list of emails \\
		\texttt{harvest.emailshttp} & Return a list of emails via HTTP \\
		\texttt{harvest.aol} & Return a list of AOL specific information \\
		\texttt{harvest.registry} & Return registry information for specific registry path \\
		\texttt{harvest.windowskeys} & Return Windows registry information \\
		\texttt{pctrl.list} & Return list of all processes \\
		\texttt{pctrl.kill} & Kill specified process set from service file \\
		\texttt{pctrl.listsvc} & Return list of all services that are running \\
		\texttt{pctrl.killsvc} & Delete/stop a specified service \\
		\texttt{pctrl.killpid} & Kill specified process \\
		\texttt{inst.asadd} & Add an autostart entry \\
		\texttt{inst.asdel} & Delete an autostart entry \\
		\texttt{inst.svcadd} & Adds a service to SCM \\
		\texttt{inst.svcdel} & Delete a service from SCM \\
		\bottomrule
	\end{tabular}
\end{table}

Tập các tính năng điều khiển máy chủ được cung cấp trong Agobot là khá đầy đủ (Bảng~\ref{table:botctrl}).
Các lệnh này bao gồm:

\begin{itemize}
\item Các lệnh để bảo mật hệ thống, ví dụ như đóng các cổng NetBIOS chia sẻ,
	vá cho các lỗ hổng phổ biến như RPC-DCOM (Blaster), v.v.
\item Nhiều lệnh để thu thập thông tin nhạy cảm bao gồm email
	địa chỉ, cdkeys, mật khẩu AOL, mật khẩu PayPal, v.v.
\item Lệnh \textbf{pctrl} để liệt kê các tiến trình đang chạy trên máy chủ và
	tiêu diệt các tiến trình cụ thể.
\item Lệnh \textbf{inst} để thêm hoặc xóa các mục tự động chạy cùng hệ thống (autostart).
\item Giám sát bàn phím và mạng dựa trên pcap (trong file \textbf{sniffer.cpp}).
\item Truy cập Windows Registry từ xa.
\item Kiểm soát hệ thống tập tin cục bộ, bao gồm khả năng tải xuống và thực thi.
\end{itemize}

\begin{exmp}
Lệnh \texttt{harvest.cdkeys} trả về một danh sách các mã kích hoạt
bằng cách định nghĩa một các game và khoá registry tương ứng chứa
mã kích hoạt game (đoạn mã~\ref{code:game_cdkeys}),
sau đó dùng hàm \texttt{CHarvest\_CDKeys::HandleCommand()}
để truy vấn và trả về kết quả (đoạn mã~\ref{code:handle_cdkeys}).
\end{exmp}

\begin{listing}[ht]
	\caption{Các game và khoá Registry tương ứng chứa mã kích hoạt (thuộc file \textbf{harvest\_cdkeys.cpp})}\label{code:game_cdkeys}
	\inputminted{cpp}{./listings/games-harvest_cdkeys.cpp}
\end{listing}

\begin{listing}[ht]
	\caption{Truy vấn mã kích hoạt qua khoá Registry và trả về kết quả nếu có (trong hàm \texttt{CHarvest\_CDKeys::HandleCommand} thuộc file \textbf{harvest\_cdkeys.cpp})}\label{code:handle_cdkeys}
	\inputminted{cpp}{./listings/handle-harvest_cdkeys.cpp}
\end{listing}

\begin{exmp}
Lệnh \texttt{pctrl.killpid} -- dừng tiến trình được chỉ định -- dùng hàm \texttt{TerminateProcess()}
để ngưng tiến trình đang thực thi (đoạn mã~\ref{code:killpid}).
\end{exmp}

\begin{listing}[ht]
	\caption{Ngừng tiến trình đang thực thi qua lệnh \texttt{pctrl.killpid} (trong file \textbf{utility.cpp}).}\label{code:killpid}
	\inputminted{cpp}{./listings/killpid-pctrl.cpp}
\end{listing}

\begin{exmp}
Lệnh \texttt{inst.add} -- thêm vào mục tự khởi động cùng Windows --
sử dụng hàm \texttt{RegCreateKeyEx()} (đoạn mã~\ref{code:add_autostart})
để tạo khoá mới và dùng hàm \texttt{RegSetValueEx()} để ghi giá trị trên hai khoá Registry
\path{HKEY_LOCAL_MACHINE\Software\Microsoft\Windows\CurrentVersion\Run}
và
\texttt{RunServices}.
\end{exmp}

\begin{listing}[ht]
	\caption{Thêm mục tự khởi động qua lệnh \texttt{inst.add} (trong file \textbf{installer.cpp}).}\label{code:add_autostart}
	\inputminted{cpp}{./listings/add_autostart.cpp}
\end{listing}

\begin{exmp}
File \textbf{keylogger.cpp}:
Theo dõi bàn phím dựa trên lớp \texttt{CKeyLog} kế thừa từ lớp đại diện luồng \texttt{CThread}.
Lớp này sử dụng các WinAPI như
\texttt{GetForegroundWindow()},
\texttt{GetWindowText()},
\texttt{GetKeyState()}
và
\texttt{GetAsyncKeyState()}
để theo dõi ký tự được gõ.
\end{exmp}

\begin{exmp}
File \textbf{utility.cpp}:
Khả năng tải xuống và thực thi dựa trên lớp \texttt{CDownloader} với các lệnh như
\texttt{http.download} (tải một file qua http),
\texttt{http.visit} (ghé thăm một trang web bất kỳ),
\texttt{http.execute} (cập nhật bot qua địa chỉ http url)
và
\texttt{http.update} (thực thi file download được từ http url).
Trong đó lệnh \texttt{http.visit} có thể được dùng để
\href{https://en.wikipedia.org/wiki/Click_fraud}{giả mạo click chuột}.
\end{exmp}

\begin{exmp}
File \textbf{nbscanner.cpp}: Cài đặt lớp \texttt{CScannerNetBios} để quét NetBIOS.
Lớp này vét cạn mật khẩu NetBIOS dựa trên một số tên người dùng và mật khẩu
thường dùng (đoạn code~\ref{code:up_netbios}).
\end{exmp}

\begin{listing}[ht]
	\caption{Một số tên và mật khẩu được dùng để vét cạn NetBIOS (trong file \textbf{nbscanner.cpp}).}\label{code:up_netbios}
	\inputminted{cpp}{./listings/upass_netbios.cpp}
\end{listing}

\begin{exmp}
Quét và khai thác lỗ hỗng RPC-DCOM nhờ vào 2 file \textbf{dcomscanner.cpp} và \textbf{dcom2scanner.cpp}.
Hai file này chứa các shellcode để thêm khai thác lỗ hỗng qua các cổng 135, 1025 và 445.
Các shellcode được hàm khai thác \texttt{CScannerDCOM2::Exploit} sử dụng để thêm người dùng,
mở cổng kết nối mới, v.v.
\end{exmp}

\section{Cơ chế lan truyền}
Lan truyền là các cơ chế được sử dụng bởi các bot để tìm kiếm các máy chủ mới.
Cơ chế lan truyền truyền thống bao gồm quét ngang đơn giản trên một cổng duy nhất
trải dài một dải địa chỉ cho trước hoặc quét dọc một địa chỉ IP duy nhất trên nhiều cổng mạng cụ thể.
Tuy nhiên, khi khả năng của botnet mở rộng, có khả năng là chúng sẽ áp dụng các phương pháp tuyên truyền tinh vi hơn

\begin{table}[ht]
	\caption{Các lệnh quét và lan truyền trong Agobot}
	\label{table:botprog}
	\centering
	\footnotesize
	\begin{tabular}{l l}
		\toprule
		\textbf{Command} &
		\textbf{Description} \\
		\midrule
		\texttt{scan.addnetrange} & Adds a network range to a bot\\
		\texttt{scan.delnetrange} & Deletes a network range from a bot\\
		\texttt{scan.listnetranges} & Returns all network ranges registered with a bot\\
		\texttt{scan.clearnetranges} & Clears all network ranges registered with a bot\\
		\texttt{scan.resetnetranges} & Resets the network ranges to the localhost\\
		\texttt{scan.enable <module name>} & Enables a scanner module e.g., DCOM\\
		\texttt{scan.disable <module name>} & Disables a scanner module\\
		\texttt{scan.startall} & Directs all bots to start scanning their network ranges\\
		\texttt{scan.stopall} & Directs all bots to stop scanning\\
		\texttt{scan.start} & Directs all enabled bots start scanning\\
		\texttt{scan.stop} & Directs all bots to stop scanning\\
		\texttt{scan.stats} & Returns results of scans\\
		\bottomrule
	\end{tabular}
\end{table}

Các cơ chế quét trong Agobot là tương đối đơn giản và không khác nhiều so với việc quét ngang và quét dọc.
Việc quét của Agobot dựa trên khái niệm về phạm vi mạng (tiền tố mạng) được cấu hình trên các bot riêng lẻ.
Khi được điều khiển, bot có thể quét trải dài một mạng hoặc chọn ngẫu nhiên các địa chỉ IP trong một dải địa chỉ.
Tuy nhiên, bộ lệnh quét hiện tại không cung cấp phương thức để phân phối một cách hiệu quả không gian địa chỉ mục tiêu trong giữa các bot với nhau. Bảng~\ref{table:botprog} tóm tắt các lệnh quét trong Agobot.

\begin{exmp}
Lệnh \texttt{scan.addnetrange} sử dụng hàm \texttt{ParseNetRange()} để thêm phạm vi mạng từ
địa chỉ IP và netmask cho trước (đoạn mã~\ref{code:netscan}).
\end{exmp}

\begin{listing}[ht]
	\caption{Hàm scan.addnetrange (trong file \textbf{scanner.cpp}).}\label{code:netscan}
	\inputminted{cpp}{./listings/netscan.cpp}
\end{listing}

\section{Phương thức khai thác và tấn công}
Phương thức khai thác lổ hỗng gồm các phương pháp cụ thể để tấn công các lỗ hổng đã biết trên các hệ thống đích.
Việc khai thác thường được kết hợp với việc quét để tìm kiếm mục tiêu.

Agobot có tập các mô đun khai thác phức tạp nhất trong số các dòng bot cùng thời (2004).
Trái ngược với các dòng bot ấy, bộ khai thác của Agobot ngày càng mở rộng thay vì trở thành các phiên bản riêng lẻ mà đi kèm với khai thác của riêng nó.
Agobot bao gồm một thư viện mạnh mẽ dựng sẵn tận dụng các khai thác (Dcom, Dameware, Radmin).
Agobot có thể tự động lan truyền qua cửa sau được cài đặt trước đó bởi Trojan (Bagle, MyDoom, v.v.)
Điều này giúp Agobot tăng cơ hội xâm nhập các máy chủ mục tiêu.
Các khai thác trong phiên bản của Agobot mà chúng tôi đánh giá bao gồm:

\begin{enumerate}
\item Bộ quét Bagle (\textbf{baglescanner.cpp}):
	quét backdoor được mở ra do các biến thể Bagle trên cổng 2745.
\item Bộ quét Dcom (\textbf{dcom2scanner.cpp} và \textbf{dcom2scanner.cpp}):
	quét lỗi tràn bộ đệm DCE-RPC nổi tiếng.
\item Bộ quét MyDoom (\textbf{doomscanner.cpp}):
	quét các backdoor được để lại bởi các biến thể của sâu MyDoom trên cổng 3127.
\item Bộ quét Dameware (\textbf{dwscanner.cpp}):
	quét các phiên bản tồn tại lỗ hỗng của công cụ quản trị mạng Dameware.
\item Bộ quét NetBIOS (\textbf{nbscanner.cpp}):
	quét vét cạn mật khẩu của NetBIOS chia sẻ.
\item Bộ quét Radmin (\textbf{radminscanner.cpp}):
	quét lỗi tràn bộ đệm Radmin.
\item Bộ quét MS-SQL (\textbf{sqlscanner.cpp}):
	quét vét cạn mật khẩu các máy chủ SQL mở.
\item Mô-đun DDoS tổng quát (\textbf{ddos.cpp}):
	cho phép bảy loại tấn công từ chối dịch vụ đối với một máy chủ mục tiêu gồm
	udp flood,
	syn flood,
	http flood,
	targa3,
	wonk flood,
	phat syn flood (?),
	icmp flood.
\end{enumerate}

Danh sách các lệnh được sử dụng để kiểm soát các cuộc tấn công này được đưa ra trong Bảng~\ref{table:botattack}.

\begin{table}[ht]
	\caption{Các lệnh tấn công DDos trong Agobot.}\label{table:botattack}
	\centering
	\footnotesize
	\begin{tabular}{l l}
		\toprule
		\textbf{Command} &
		\textbf{Description} \\
		\midrule
\texttt{ddos.udpflood} & Starts a UDP flood\\
\texttt{ddos.synflood} & Starts a SYN flood\\
\texttt{ddos.httpflood} & Starts an HTTP flood\\
\texttt{ddos.phatsyn} & Starts a PHAT SYN flood\\
\texttt{ddos.phaticmp} & Starts PHAT ICMP flood\\
\texttt{ddos.phatwonk} & Starts PHATwonk flood\\
\texttt{ddos.targa3} & Start a targa3 flood\\
\texttt{ddos.stop} & Stops all floods\\
		\bottomrule
	\end{tabular}
\end{table}

Trong đó đối với lệnh tấn công \texttt{ddos.phatsyn} 
(file \textbf{ddos.cpp} và \textbf{wonk.cpp}), 
lệnh sử dụng lớp \texttt{CDDOSPhatWonkFlood}
với hàm \texttt{SendPhatWonk()} quét \textbf{28} cổng để tìm cổng đang mở 
và gửi 1 gói SYN theo sau là 1023 gói ACK để làm ngập 
địa chỉ IP đích một cách liên tục (đoạn mã~\ref{code:phatwonk}).

\begin{listing}[ht]
	\caption{Hàm \texttt{SendPhatWonk()} được gọi bởi lệnh \texttt{ddos.phatsyn} (trong file \textbf{wonk.cpp}).}\label{code:phatwonk}
	\inputminted{cpp}{./listings/phatwonk.cpp}
\end{listing}

\section{Phương thức phân phối malware}
Các \gls{packer} (trình đóng gói) và \gls{encoder} (trình mã hóa shellcode) từ lâu đã được sử dụng
trong phân phối phần mềm hợp pháp để nén và làm rối mã máy nhằm chống dịch ngược.

Agobot áp dụng kỹ thuật tương tự vì những lý do đó.
Tuy nhiên, Agobot tách mã khai thác ra khỏi phần phân phối:

\begin{itemize}
\item Mã khai thác được sử dụng để mở một remote shell dựa trên khai thác
một lỗ hỗng (ví dụ như lỗi tràn bộ đệm)
\item Từ shell, phần tập tin nhị phân độc hại đã mã hoá sau đó được tải lên bằng HTTP hoặc FTP.
\end{itemize}

Sự tách rời này cho phép bộ mã hoá (encoder) được sử dùng đồng thời trên
nhiều mã khai thác do đó hợp lý hoá mã nguồn
và có khả năng đa dạng hóa các dòng bit của kết quả.

Agobot sử dụng một \gls{encoder} để làm rối \gls{assembly} và loại bỏ các byte null
(mà kết thúc một chuỗi trong ngôn ngữ lập trình C) ra khỏi một chuỗi các lệnh x86.
Như có thể thấy trong đoạn~\ref{code:encoder}, 
đoạn mã sử dụng khoá \textbf{XOR} với giá trị 0x98 rồi tiếp tục kiểm tra liệu kết quả xor với 
shellcode còn chứa các kí tự có mã ASCII 0x5C, 0x00, 0x0A hoặc 0x0D.
Nếu sai, nó tiếp tục thử các giá trị tiếp theo (0x99, 0x9a, \ldots) cho đến khi tìm được giá trị thích hợp.
Giá trị tìm được này sẽ được sao chép vào shellcode tại vị trí \texttt{ENCODER\_OFFSET\_XORKEY}.

Biện pháp này đánh bại hoặc ít nhất là làm phức tạp đáng kể việc phát hiện mã độc dựa trên chữ ký (MD5, SHA-1), \dots

\begin{listing}[ht]
\caption{Hàm mã hoá shellcode của Agobot (thuộc file \textbf{shellcode.cpp})}\label{code:encoder}
\inputminted[
	frame=lines,
	framesep=2mm,
	baselinestretch=1.2,
	fontsize=\scriptsize
]{cpp}{./listings/shellcode.cpp}
\end{listing}

\section{Phương thức obfuscation}
\Gls{obfuscation} là cơ chế được sử dụng để ẩn các chi tiết của những gì được
truyền qua mạng và những gì đi đến để thực hiện trên máy chủ đầu cuối.
Agobot cung cấp việc \gls{obfuscation} một cách hạn chế.

Agobot hỗ trợ 4 kịch bản mã hóa đa hình khác nhau:

\begin{itemize}
\item \verb|POLY_TYPE_XOR| (mã hoá bằng phép XOR),
\item \verb|POLY_TYPE_SWAP| (hoán đổi các byte liên tiếp),
\item \verb|POLY_TYPE_ROR| (xoay phải),
\item \verb|POLY_TYPE_ROL| (xoay trái).
\end{itemize}

Trong đó, hàm \texttt{DoPolymorph} (đoạn mã~\ref{code:poly}) (trong \textbf{polymorph.cpp}) có vai trò mã hoá đa hình file.
Hàm này ánh xạ tập tin, tìm kiếm nơi phù hợp cho encoder và phần code,
đặt encoder vào trong phần phù hợp đó,
thiếp lập entrypoint (điểm bắt đầu của tệp thực thi)
trỏ đến encoder,
thiết lập cờ trong header để nhận diện được tệp đã mã hoá
và cuối cùng lưu file xuống đĩa.
và phần mã, tạo bộ mã hóa vào phần,
đặt điểm vào trong tệp cho bộ mã hóa, đặt cờ trong
tiêu đề để phát hiện các tệp đã được mã hóa và cuối cùng lưu tệp
quay lại đĩa lưu trữ. 
Mục đích là biến đổi file mỗi lần nó được lây lan để tránh né các chương trình chống vi-rút,
khiến cho các chương trình chống vi-rút phải lây lan nó 
mỗi lần chúng tìm thấy phiên bản mới của các file bị biến đổi.

\newenvironment{code}{\captionsetup{type=listing}}{}
\begin{code}
\caption{Hàm đa hình \texttt{DoPolymorph()} (trong \textbf{polymorph.cpp})}\label{code:poly}
\inputminted[
	frame=single,
	framesep=2mm,
	baselinestretch=1.2,
	fontsize=\tiny
]{cpp}{./listings/poly.cpp}
\end{code}

\section{Phương thức deception}

\Gls{deception} đề cập đến các cơ chế được sử dụng để tránh bị phát hiện khi bot được cài đặt
trên máy chủ đích (cơ chế này cũng được gọi là \texttt{rootkit}), bao gồm:

\begin{itemize}
\item Biện pháp để ẩn các tiến trình và tệp để người dùng không nhìn thấy.
\item Các biện pháp kiểm tra và chống gỡ lỗi của các chương trình như 
	OllyDebug, 
	SoftIce 
	và 
	procdump.
\item Kiểm tra ảo hoá VMWare nhằm cung cấp thông tin sai lệnh khi phân tích mã độc này.
\item Tiêu diện các tiến trình của các ứng dụng chống virus phổ biến thông qua cách chèn mã độc.
\item Thay đổi danh mục DNS của máy chủ để khiến cho địa chỉ của các trang web chống vi-rút trỏ đến localhost.
\end{itemize}

Tất cả các cơ chế này phần lớp được cài đặt ở trong tập \textbf{utility.cpp}.
Trong đó có các hàm 
\texttt{IsVMWare()} (đoạn mã~\ref{code:vmware})để kiểm tra có phải chạy trong máy ảo VMWare,
hàm \texttt{AddHosts()} (đoạn mã~\ref{code:etc_hots}) thêm danh mục các trang web chống vi-rút trỏ đến localhost,
hàm \texttt{IsODBGLoaded(), 
IsSICELoaded(), 
IsSoftIceNTLoaded()
và
IsBPX()} (đoạn mã~\ref{code:isdebug}) kiểm tra trình gỡ lỗi 
OllyDbg, SoftIce
và kiểm tra breakpoint~\dots~
Các hàm trên đều được gọi qua đối tượng của lớp \texttt{CDebugDetect}.

\begin{code}
	\caption{Hàm phát hiện máy ảo VMWare (trong \textbf{utility.cpp})}\label{code:vmware}
	\inputminted[
	frame=single,
	framesep=2mm,
	baselinestretch=1.2,
	fontsize=\tiny
	]{cpp}{./listings/vmware.cpp}
\end{code}

\begin{code}
	\caption{Hàm thêm mục vào file HOSTS  (trong \textbf{utility.cpp})}\label{code:etc_hots}
	\inputminted[
	frame=single,
	framesep=2mm,
	baselinestretch=1.2,
	fontsize=\tiny
	]{cpp}{./listings/add_hosts.cpp}
\end{code}

\begin{code}
	\caption{Hàm phát hiện gỡ lỗi (trong \textbf{utility.cpp})}\label{code:isdebug}
	\inputminted[
	frame=single,
	framesep=2mm,
	baselinestretch=1.2,
	fontsize=\tiny
	]{cpp}{./listings/isdebug.cpp}
\end{code}

Khi mà các biện pháp này được cải thiện thêm, 
việc phát hiện hệ thống bị xâm nhập sẽ trở nên khó khăn hơn và làm phức tạp
nhiệm vụ của hệ thống nhận diện rootkit và chương trình chống virus.

% \section{Kết chương}

