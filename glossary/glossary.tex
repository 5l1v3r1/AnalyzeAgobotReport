% Danh mục thuật ngữ
\newglossaryentry{codebase}
{
	name={codebase},
	description={toàn bộ tập mã nguồn được sử dụng để xây dựng một hệ thống phần mềm, ứng dụng hoặc thành phần phần mềm cụ thể},
	user1={https://en.wikipedia.org/wiki/Codebase}
}

\newglossaryentry{malware}
{
	name={malware},
	description={phần mềm ác tính, phần mềm độc hại, phần mềm gây hại hay mã độc},
	user1={https://vi.wikipedia.org/wiki/Ph\%E1\%BA\%A7n_m\%E1\%BB\%81m_\%C3\%A1c_\%C3\%BD}
}

\newglossaryentry{obfuscation}
{
	name={obfuscation},
	description={cơ chế làm rối mã nguồn để các công cụ tự động không phát hiện phần mềm độc hại},
	user1={https://en.wikipedia.org/wiki/Obfuscation_\(software\)}
}

\newglossaryentry{deception}
{
	name={deception},
	description={cơ chế được sử dụng để tránh phát hiện khi bot được cài đặt trên một máy chủ}
}

\newglossaryentry{assembly}
{
	name={assembly},
	description={hợp ngữ: một ngôn ngữ lập trình bậc thấp dùng để viết các chương trình máy tính}
}

\newglossaryentry{packer}
{
	name={packer},
	description={trình đóng gói dùng để nén tệp thực thi bất kỳ, sau đó kết hợp dữ liệu nén với phần mã dùng để giải nén vào một tệp thực thi kết quả duy nhất}
}

\newglossaryentry{encoder}
{
	name={shellcode encoder},
	description={chương trình xử lý shellcode để shellcode có thể được truyền một cách bình thường, có thể bao gồm cơ chế ngược để nhận kết quả gốc ở nơi nhận}
}

\newglossaryentry{shellcode}
{
	name={shellcode},
	description={một đoạn mã nhỏ được trong việc khai thác lỗ hổng phần mềm, nó được gọi là ``shellcode'' do nó thường mở một chương trình xử lý dòng lệnh để kẻ tấn công có thể điều khiển máy bị xâm nhập}
}

%% Danh mục từ viết tắt
%% Ví dụ
%\newacronym{http}{HTTP}{HyperText Transfer Protocol}
%\newacronym{ai}{AI}{Artificial Intelligence}
%\newacronym{cart}{CART}{Classification and Regression Trees}
%\newacronym{csdl}{CSDL}{Cơ Sở Dữ Liệu}
%\newacronym{xss}{XSS}{Cross-Site Scripting}
%\newacronym{id3}{ID3}{Iterative Dichotomiser 3}
%\newacronym{csic}{CSIC}{Consejo Superior de Investigaciones Científicas}
